\documentclass{article}
\title{Assignment 3: A Machine for Sorting Bricks}
\author{Michael Carpenter \\ Benjamin Liebersohn}
\date{\today}
\begin{document}
\maketitle

\section{Introduction}

\section{Goals}

\section{Hardware}
The Lego sorter is designed to sort the Lego bricks into two categories: light, and dark. The way our sorter determines this quality is through a reflectivity sensor which acts as a proxy to measuring shade. The idea behind this is that a red LED shines light onto a brick, which then reflects back into a light sensor. If a brick has a lighter shade, it absorbs less of the light, which can then reflect and be measured. A higher reading, thus means our brick is likely a lighter shade because it is more reflective of red light. In order to decide what constitutes light, or dark, our sorter begins by taking samples of light readings and aggregating them to find the ambient light sensor reading. Once it has done this, it can begin to determine if a brick is light or dark. Once that determination has been made, the motor powers a lever which pushes the brick out of the sorter in one of two directions: either the lever is turned counter clockwise to push a light brick to the left (with the light sensor facing the user) or counter clockwise to push the dark brick to the right. From there, a bumper and deflector guide the brick out of the sorter, into the cups below. Occasionally, bricks may jam in between a rail and sensor, which is corrected by gears we allow to slip, and thus repeatedly tap the brick with the lever. Since the bricks only jam in front of a sensor, the sensor can detect when there is still a brick on the platform, allowing the motor to continue to tap on the jammed brick, which is normally enough to coax the brick out of the corner.
\section{Software}

\section{Discussion}

In a future design, we would like to use slightly different methods to measure reflectivity. Because the LED we used emits a red spectrum light, some colors of Lego bricks reflect light back in unexpected ways. Instead, using the white incandescent light supplied with the kit would give a more balanced spectrum to reflect back to the light sensor, which would give us more accurate readings. Another change to make to the methods involving the light sensor would be to continue to sample ambient readings throughout the sorting process. Instead of relying on only an initial sampling, additional samples of ambient light could be taken when bricks are not in front of the sensor. The sorter could know if a brick is being examined by using the ultrasonic sensors to determine if something is directly in front of the light sensor. This way, additional samples of ambient light are taken, allowing the ambient light reading to reflect the lighting in the room in a "hot-swapping" manner. Another aspect of our design which we could improve further would be our accuracy at landing bricks in the cup. While most bricks do land in the correct cup, sometimes a brick might bounce back out, or get pushed back towards the motor. The best solution to these problems will likely involve looking for ways to lower the sorting platform and arm, in order to lessen the likelihood of failure due to the bricks bouncing or falling inaccurately.  
\end{document}
